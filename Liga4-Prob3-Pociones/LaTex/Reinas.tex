% !TeX spellcheck = es_ES
\documentclass{scrartcl}
%\documentclass{book}
\usepackage[spanish]{babel}
\usepackage[utf8]{inputenc}
\usepackage{enumerate}
\usepackage{graphicx}
\usepackage{color}
\usepackage{float}
\usepackage{multicol}
\usepackage{lscape}
\usepackage{fancyhdr}
\usepackage{amssymb}
\usepackage{textcomp}
\usepackage{amsmath}
\usepackage{tabularx}
\usepackage[hidelinks,colorlinks=true, linkcolor=black]{hyperref} %hyperref para hacer los links del indice y usar \url{URL} y \href{URL}{text}y hidelinks para que no se rodeen con una caja de color los enlaces. Más información en: http://en.wikibooks.org/wiki/LaTeX/Hyperlinks




\pagestyle{fancy}
%\usepackage{tabulary}
%\usepackage{/home/manolito/IISSI/pruebaLaTex/comfortaa/tex/latex/comfortaa/comfortaa}

%\lfoot[\thepage]{\bfseries{\includegraphics[width=0.020\textwidth]{imagenes/logo.png} PhotoLoad}}
\cfoot[\thepage]{}
\rfoot[\thepage]{\thepage}
%\lfoot[c1]{GECH}
%\lfoot[e1]{GECH}
\lhead[\thepage]{}
\rhead[\thepage]{}
\renewcommand{\headrulewidth}{0.0pt}
\renewcommand{\footrulewidth}{0.2pt}
\hyphenation{se-ma-nal per-so-na-les ope-ra-tivo mo-ni-tor mues-tra pide correc-ta-men-te}



%\renewcommand{\baselinestretch}{1.5} %interlineado

\begin{document}
\begin{titlepage}
\begin{center}
	\vspace*{-1in}
	\vspace*{2in}
	\begin{Huge}
		%si lo quieres poner mas grande Huge, huge, uno mas pequeño LARGE, dos más pequeños Large
		\textbf{ADDA} \\
	\end{Huge}
\end{center}
\end{titlepage}
\section{Introducción}
\begin{tabularx}{\textwidth}{|r|X|}
	\hline
	\multicolumn{2}{|c|}{\textbf{Problema de Pociones}} \\
	\hline
	Técnica & \textbf{Programación Dinámica} \\
	\hline
	Tamaño & \textbf{despensa.size() - i} 
	\\
	\hline
	Propiedades Compartidas & \begin{enumerate}[{\textbf{\textperiodcentered}}]
		\item List\textless Pocion\textgreater   \textbf{despensa}: Guarda todas las pociones
		\item Integer \textbf{nIni}: nivel inicial oponente
		\item TipoPersonaje	\textbf{tp}: dirá si es nigromante o no
	\end{enumerate}
	\\
	\hline
	Propiedades Individuales & 
	\begin{enumerate}[\textbf{\textperiodcentered}]
		\item Integer \textbf{i} en [0, despensa.size]: posición actual de la lista
		\item Integer \textbf{nAct}: nivel actual
		\item Integer \textbf{cAcu}: coste acumulado
	\end{enumerate} \\
	\hline
	Solución & \textbf{s = (as, ct, dt)} ; siendo:
	
	\textbf{\textperiodcentered} as: MultiSet de alternativas seleccionadas en el paso
	
	\textbf{\textperiodcentered} ct: coste total
	
	\textbf{\textperiodcentered} dt: daño total\\
	\hline
	Objetivo & Encontrar \textgravedbl s\textacutedbl tal que \textgravedbl dt \textgreater =  nIni\textacutedbl y \textgravedbl ct\textacutedbl  tenga el menor valor posible\\
	\hline
	Solución Parcial & \textbf{Sp = (a, ct)} \\
	\hline
	Alternativas & A\textsubscript{cAcu, i, despensa} = \{a:k...0\};
	
	k = min(can, {nAct}/{can});
	
	can = despensa.get(i).getCant();
	
	Falta filtro de si es nigromante o no\\
	\hline
	Instanciación & pp(despensa, nAct, tipo) = ppg(despensa, nIni, tipo, 0, nIni,0) \\
	\hline
	Problema Generalizado & \[ ppg(X, p) =
	\begin{cases}
	(null, 0)			& \quad \text{i \textgreater  d.length() or nAct \textless= 0}\\
	cA_{a\epsilon A_{co, i, d}}  (c(X, a, ppg(sp))) & \quad \text{ecoc}\\
	\end{cases}
	\] 
	co=cAcu; d=despensa;
	
	p=(i, nAct, cAcu)
	
	X=(despensa, nIni, tipo)
	
	sp=(i++,d.get(i).getDano()*a+nAct, d.get(i).getCoste()*a)\\
	\hline
	sA & (X, a)=(a, cAcu+pociones.get(i).getCoste()*a) \\
	\hline
	Solución reconstruida & ?¿ \\
	\hline
	
\end{tabularx}
\end{document}
